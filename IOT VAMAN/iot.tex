\def\mytitle{IOT ASSIGMNMENT}
\def\myauthor{BOLLA VAMSIKRISHNA}
\def\contact{bollavamsi04@gmail.com}
\def\mymodule{Future Wireless Communications (FWC)}
\documentclass[journal,12pt,twocolumn]{IEEEtran}

\usepackage{setspace}
\usepackage{gensymb}
\usepackage{xcolor}
\usepackage{caption}
\usepackage[hyphens,spaces,obeyspaces]{url}
\usepackage[cmex10]{amsmath}
\usepackage{mathtools}
\singlespacing
\usepackage{amsthm}
\usepackage{mathrsfs}
\usepackage{txfonts}
\usepackage{stfloats}
\usepackage{cite}
\usepackage{cases}
\usepackage{subfig}
\usepackage{longtable}
\usepackage{multirow}
\twocolumn


\usepackage{graphicx}
\graphicspath{{./images/}}
\usepackage[colorlinks,linkcolor={black},citecolor={blue!80!black},urlcolor={blue!80!black}]{hyperref}
\usepackage[parfill]{parskip}
\usepackage{lmodern}
\usepackage{tikz}
\usepackage{circuitikz}
\usepackage{karnaugh-map}
\usepackage{pgf}
\usepackage[hyphenbreaks]{breakurl}

\usepackage{tabularx}
\usetikzlibrary{calc}

\renewcommand*\familydefault{\sfdefault}
\usepackage{watermark}
\usepackage{lipsum}
\usepackage{xcolor}
\usepackage{listings}
\usepackage{float}
\usepackage{titlesec}
\DeclareMathOperator*{\Res}{Res}
\renewcommand\thesection{\arabic{section}}
\renewcommand\thesubsection{\thesection.\arabic{subsection}}
\renewcommand\thesubsubsection{\thesubsection.\arabic{subsubsection}}

\renewcommand\thesectiondis{\arabic{section}}
\renewcommand\thesubsectiondis{\thesectiondis.\arabic{subsection}}
\renewcommand\thesubsubsectiondis{\thesubsectiondis.\arabic{subsubsection}}
\titlespacing{\subsection}{1pt}{\parskip}{3pt}
\titlespacing{\subsubsection}{0pt}{\parskip}{-\parskip}
\titlespacing{\paragraph}{0pt}{\parskip}{\parskip}
\newcommand{\figuremacro}[5]{
    \begin{figure}[#1]
        \centering
        \includegraphics[width=#5\columnwidth]{#2}
        \caption[#3]{\textbf{#3}#4}
        \label{fig:#2}
    \end{figure}
}

\lstset{
frame=single, 
breaklines=true,
columns=fullflexible
}

%\thiswatermark{\centering \put(400,-128.0){\includegraphics[scale=0.3]{logo}} }
\title{\mytitle}
\author{\myauthor\hspace{1em}\\\contact\\IITH\hspace{0.5em}-\hspace{0.6em}\mymodule}
\date{20-12-2022}
\def\inputGnumericTable{}                                 %%
\lstset{
%language=C,
frame=single, 
breaklines=true,
columns=fullflexible
}
 \begin{document}
%

\theoremstyle{definition}
\newtheorem{theorem}{Theorem}[section]
\newtheorem{problem}{Problem}
\newtheorem{proposition}{Proposition}[section]
\newtheorem{lemma}{Lemma}[section]
\newtheorem{corollary}[theorem]{Corollary}
\newtheorem{example}{Example}[section]
\newtheorem{definition}{Definition}[section]
%\newtheorem{algorithm}{Algorithm}[section]
%\newtheorem{cor}{Corollary}
\newcommand{\BEQA}{\begin{eqnarray}}
\newcommand{\EEQA}{\end{eqnarray}}
\newcommand{\define}{\stackrel{\triangle}{=}}
\bibliographystyle{IEEEtran}

\vspace{3cm}
\maketitle
\tableofcontents
  \section{QUESTION}
     An 8085 microprocessor acesses two memory locations $(2001H)$ and $(2002H)$, that contains $8$-bit numbers $98H$ and $B1H$ , respectively. the following program is executed:\\
     LXI H,$2001H$\\
     MVI A,$21H$\\
     INX H \\
     ADD M \\
     INX H \\
     MOV M,A \\
     HLT \\
     At the end of this program ,the memory location $2003H$ contains the number in decimal(base $10$ ) form\\
     \section{COMPONENTS}
     \begin{tabularx}{0.46\textwidth} { 
  | >{\centering\arraybackslash}X 
  | >{\centering\arraybackslash}X 
  | >{\centering\arraybackslash}X
  | >{\centering\arraybackslash}X | }
\hline
\textbf{Component}& \textbf{Values} & \textbf{Quantity}\\
\hline
Vaman & LC & 1 \\  
\hline
JumperWires & M-F & 20 \\ 
\hline
Breadboard & &1 \\
\hline
LCD & &1\\
\hline
Resistor & 220ohms & 2 \\
\hline
\end{tabularx}
\section{LCD CONNECTIONS}

 \begin{figure}[H]
\centering
\includegraphics[width=\columnwidth]{figs/lcd1.jpg}
\caption{lcd1}
\label{fig:lcd1.jpg}
\end{figure}
     





\section{IMPLEMENTATION}
  \begin{tabularx}{0.4\textwidth} { 
  | >{\centering\arraybackslash}X 
  | >{\centering\arraybackslash}X  | }


\hline
\textbf{Vaman PIN} & \textbf{lcd } \\ 
%\hline
%GND & 1 \\
%\hline
%5V & 2 \\
%\hline
%GND & 3 \
\hline
GPIO19 & 4 \\
%\hline
%GND & 5 \\
\hline
GPIO23 & 6\\
\hline
GPIO18 & 11\\
\hline
GPIO17 & 12 \\
\hline
GPIO16 & 13 \\
\hline
GPIO15 & 14 \\
%\hline
%5V & 15 \\
%\hline
%GND & 16 \\
\hline
\end{tabularx}

\begin{center}
    Connections
\end{center}
\paragraph{PROCEDURE}
    
    1. Connect the circuit as per the above table.\\
    2. connect the lcd to vaman\\
\\ \begin{tabularx}{0.45\textwidth} { 
  | >{\centering\arraybackslash}X |}
  \hline
        https://github.com/Vamsichowdary04/Future\\\-Wireless\-Communication-FWC\\\-/blob/main/IOTVAMAN/M210.cpp\\
  \hline
\end{tabularx}
    \section{LCD OUTPUT}

 \begin{figure}[H]
\centering
\includegraphics[width=\columnwidth]{figs/lcd2.jpg}
\caption{vaman connection with lcd}
\label{fig:lcd2.jpg}
\end{figure}

 \bibliographystyle{ieeetr}
\end{document}
