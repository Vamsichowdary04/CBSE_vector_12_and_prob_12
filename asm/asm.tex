\def\mytitle{ASM ASSIGNMENT}
\def\myauthor{BOLLA VAMSIKRISHNA}
\def\contact{bollavamsi04@gmail.com}
\def\mymodule{Future Wireless Communications (FWC)}
\documentclass[journal,12pt,twocolumn]{IEEEtran}

\usepackage{tabularx}
\usepackage{float}
\usepackage{graphicx}
\usepackage{siunitx}
\usepackage{inputenc}
\usepackage{microtype}
\usepackage{multirow}
\twocolumn
\usepackage{etoolbox}
\tracingpatches

\usepackage{amsmath}
\usepackage{microtype}



%\thiswatermark{\centering \put(400,-128.0){\includegraphics[scale=0.3]{logo}} }
\title{\mytitle}
\author{\myauthor\hspace{1em}\\\contact\\IITH\hspace{0.5em}-\hspace{0.6em}\mymodule}
\date{20-12-2022}
\def\inputGnumericTable{}                                 %%




 

\begin{document}

\newtheorem{theorem}{Theorem}[section]
\newtheorem{problem}{Problem}
\newtheorem{proposition}{Proposition}[section]
\newtheorem{lemma}{Lemma}[section]
\newtheorem{corollary}[theorem]{Corollary}
\newtheorem{example}{Example}[section]
\newtheorem{definition}{Definition}[section]
%\newtheorem{algorithm}{Algorithm}[section]
%\newtheorem{cor}{Corollary}
\newcommand{\BEQA}{\begin{eqnarray}}
\newcommand{\EEQA}{\end{eqnarray}}
\newcommand{\define}{\stackrel{\triangle}{=}}
\bibliographystyle{IEEEtran}

\vspace{3cm}
  \maketitle
  \tableofcontents






 
\section{QUESTION} 
 Devolop a Nand gate ($\overline{A.B}$) or ($ \overline{A}+\overline{B} $) using 2x1 MUX , verify its output using an LED.

 \section{COMPONENTS}
  \begin{tabularx}{0.45\textwidth} { 
  | >{\centering\arraybackslash}X 
  | >{\centering\arraybackslash}X 
  | >{\centering\arraybackslash}X
  | >{\centering\arraybackslash}X | }
\hline
 \textbf{Component}& \textbf{Values} & \textbf{Quantity}\\
\hline
Arduino & UNO & 1 \\  
\hline
JumperWires& M-M & 2\\ 
\hline
Breadboard &  & 1 \\
\hline
LED & &1 \\
\hline
Resistor &220ohms & 1\\
\hline
\end{tabularx}

\begin{center}
Table.COMPONENTS
\end{center}

\section{CIRCUIT DIAGRAM}
\begin{figure}[H]
\centering
\includegraphics[width=0.35\textwidth]{figs/mux.jpg}
\caption{NAND GATE USING 2X1 MUX}
\label{fig:mux.jpg}
\end{figure}
\section{PROCEDURE}
\begin{enumerate}
\item Connect the anode (longer leg) of the LED to digital pin 13 (PB5) on the Arduino Uno.
\item Connect the cathode (shorter leg) of the LED to a current-limiting resistor (e.g., 220 ohms).
\item Connect the other end of the current-limiting resistor to the GND (ground) pin on the Arduino Uno.
\end{enumerate}
\section{TRUTH TABLE}
\begin{figure}[H]
\centering
\includegraphics[width=0.4\textwidth]{figs/nand.jpg}
\caption{NAND GATE TRUTH TABLE}
\label{fig:nand.jpg}
\end{figure}
\subsection{LOGIC}
From the Circuit diagram we get
\begin{align}
	F&=\overline{A}+A\cdot\overline{B} 
\end{align}
Using Redundancy law
\begin{align}
	F&=\overline{A}+\overline{B}
\end{align}
which can also be written as
\begin{align}
    F&=\overline{A.B}
\end{align}
Which is the logic of a NAND gate
\section{LED OUTPUT}
\begin{figure}[H]
\centering
\includegraphics[width=0.4\textwidth]{figs/asm cir.jpg}
\caption{LED OUTPUT}
\label{fig:asm cir.jpg}
\end{figure}

\section{OSCILLOSCOPE VISUALIZATION}
\begin{figure}[H]
\centering
\includegraphics[width=0.4\textwidth]{figs/asm osclp.jpg}
\caption{OSCILLOSCOPE VISUALIZATION}
\label{fig:asm osclp.jpg}
\end{figure}
\section{CONCLUSION}

Hence we have implemented the NAND gate using 2X1 MUX digital circuit.Execute the circuit using below code.
   \begin{tabularx}{0.46\textwidth} { 
  | >{\centering\arraybackslash}X |}
  \hline
https://github.com/Vamsichowdary04/CBSE\\\_vector\_12\_and\_prob\_12/blob/main/asm\\/mux.asm\\
  \hline
\end{tabularx}
\bibliographystyle{ieeetr}
\end{document}
